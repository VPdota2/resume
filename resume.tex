% !TEX program = xelatex

\documentclass{resume}
%\usepackage{zh_CN-Adobefonts_external} % Simplified Chinese Support using external fonts (./fonts/zh_CN-Adobe/)
%\usepackage{zh_CN-Adobefonts_internal} % Simplified Chinese Support using system fonts

\begin{document}
\pagenumbering{gobble} % suppress displaying page number

\name{Sheng Yuan}

\basicInfo{
  \email{sheng.yuan@uky.edu} \textperiodcentered\ 
  \phone{(+1) 859-285-5956} \ }

\section{\faGraduationCap\ Education}
\datedsubsection{\textbf{University of Kentucky}, Lexington, U.S.}{2016 -- Present}
\textit{Ph.D.} in Statistics, expected May 2021
\datedsubsection{\textbf{Nankai University}, Tianjin, China}{2011 -- 2015}
\textit{B.S.} in Statistics

\section{\faCogs\ Skills}
\begin{itemize}[parsep=0.5ex]
  \item Programming Languages: R,SAS,JMP,Python,SQL
  \item Data Skill: Big data analysis, Machine Learning, Deep Learning
\end{itemize}


\section{\faUsers\ Projects}
\datedsubsection{\textbf{Adding variability to Deep Neural Network}}{ July 2019 -- Present}
\role{Individual Project}{Using python, keras, pytorch, tensorflow and Linux}
\begin{itemize}
 \item Applying Variability method (variational inference, etc.) to add variability on the parameters of deep neural networks(using pytorch and keras in python), build 'confidence interval'-liked statistical concept in dnn. 
  \item Using Bayesian method in deep learning to avoid overfitting, and compare with other method like dropout or model pruning.
\end{itemize}

\datedsubsection{\textbf{111: Assessment Of Adult Non-trauma Icu Transfer Requests In Kentucky Using Publicly Available Data}}{May. 2018 -- Jan. 2019}
\role{Second Author}{Collabrate with Audrey Johnson}
Accepted by Critical Care Medicine: January 2019 - Volume 47 - Issue 1 - p 38
\begin{itemize}
  \item Data is HCUP data in KY (2011–16). Use the expected value in contigency table to analyze two categorical variables, and showed the trend. Using partial F test for quantitative data based on categorical variable, showed statistical significance.
  \item Compare critical illness(sepsis) to the whole dataset, showed sepsis patients that they will transferred to higher case volume hospitals, and had lower mortality rate. (confidence interval for sepsis patients far higher)
\end{itemize}

\datedsubsection{\textbf{Associations between changes in olfactory perception and cognitive symptoms in mild traumatic brain injury}}{May. 2018 -- Present}
\role{Second Author}{Collabrate with Dong Han}
Submitted to Clinical Neurology and Neurosurgery, Jan.2019
\begin{itemize}
   \item Use JMP to do categorical data analysis (Fisher's exact test) on variables may be connected to olfactory perception's changes in brain injury, report p-value and find expressive difficulties.
  \item Doing logistic regression by SAS's backward-forward selection method and show results that like male are 24\% more likely to be in olfactory deficits and so on.
\end{itemize}

\datedsubsection{\textbf{Project: Determinants of Plasma Retinol and Beta-Carotene Levels
}}{March. 2017 -- May. 2017}
\role{Collabrate with Tiantian Zeng and Liyu Gong}{}
\begin{itemize}
  \item In class project to learn multivariate linear regression by using R as well as github \& cran's build in function.
  \item Compared our final model to CMU's model, also $R^{2}$ is enhanced from 0.2254 to 0.2802 with less selected variables, which is a great enhancement also avoid overfitting.
\end{itemize}

% Reference Test
%\datedsubsection{\textbf{Paper Title\cite{zaharia2012resilient}}}{May. 2015}
%An xxx optimized for xxx\cite{verma2015large}
%\begin{itemize}
%  \item main contribution
%\end{itemize}


%% Reference
%\newpage
%\bibliographystyle{IEEETran}
%\bibliography{mycite}
\end{document}
