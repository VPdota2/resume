% !TEX program = xelatex

\documentclass{resume}
%\usepackage{zh_CN-Adobefonts_external} % Simplified Chinese Support using external fonts (./fonts/zh_CN-Adobe/)
%\usepackage{zh_CN-Adobefonts_internal} % Simplified Chinese Support using system fonts

\begin{document}
\pagenumbering{gobble} % suppress displaying page number

\name{Sheng Yuan}

\basicInfo{
  \email{sheng.yuan@uky.edu} \textperiodcentered\ 
  \phone{(+1) 859-285-5956} \ }

\section{\faGraduationCap\ Education}
\datedsubsection{\textbf{University of Kentucky}, Lexington, U.S.}{2016 -- Present}
\textit{Ph.D.} in Statistics, expected May 2021
\datedsubsection{\textbf{Nankai University}, Tianjin, China}{2011 -- 2015}
\textit{B.S.} in Statistics

\section{\faCogs\ Skills}
\begin{itemize}[parsep=0.5ex]
  \item Programming Languages: R,SAS,JMP,Python,SQL
  \item Data Skill: Big data analysis, Machine Learning, Deep Learning
\end{itemize}


\section{\faUsers\ Projects}
\datedsubsection{\textbf{Adding variability to Deep Neural Network}}{ July 2019 -- Present}
\role{Individual Project}{Using python, keras, pytorch, tensorflow and Linux}
\begin{itemize}
 \item Use Pytorch functional API in Python to apply Variability method (variational inference, etc.) to deep neural networks.
 \item Use Keras functional API in Python to implement Bayesian method on neural network, to achieve better performance than traditional variational method on cifar10 dataset.
\end{itemize}

\datedsubsection{\textbf{111: Assessment Of Adult Non-trauma Icu Transfer Requests In Kentucky Using Publicly Available Data}}{May. 2018 -- Present}
\role{Second Author}{Collabrate with Audrey Johnson}
Accepted by Critical Care Medicine: January 2019 - Volume 47 - Issue 1 - p 38
\begin{itemize}
  \item Use JMP on HCUP data in KY to apply hypothesis test on data, showed statistical significance by p-value and data visualization. 
  \item Use SAS to compare critical illness(sepsis) to the whole dataset, extract feature from the dataset, and do classification by SVM. 
\end{itemize}

\datedsubsection{\textbf{Associations between changes in olfactory perception and cognitive symptoms in mild traumatic brain injury}}{May. 2018 -- Present}
\role{Second Author}{Collabrate with Dong Han}
Submitted to Clinical Neurology and Neurosurgery, Jan.2019
\begin{itemize}
   \item Use JMP to do hypothesis test to select variables related to brain injury.
  \item Use SAS to do logistic regression by backward-forward selection method, interpret the parameters.
  \item Use sklearn in python to implement knn classification for critical illness.
\end{itemize}

\datedsubsection{\textbf{Project: Nonlinear Dimension Reduction and Manifold Learning}}{March. 2018 -- May. 2018}
\role{Personal Project}{}
\begin{itemize}
  \item Arbitrarily build Swiss-Roll maniforld, and Isomap especillaly Landmark Isomap did the best in classfy large dimensional data compare with PCA.
  \item Use Matlab to implement Isomap, LIsomap, LLE, Laplacian Eigenmaps and Hessian LLE to do dimensional reduction on COIL20 data, and compare result with PCA and KPCA, while classification method is LDA, QDA and kNN, find that PCA is best in dimensional reduction in this method while Isomap is the next.  
  
\end{itemize}

\datedsubsection{\textbf{Project: Determinants of Plasma Retinol and Beta-Carotene Levels
}}{March. 2017 -- May. 2017}
\role{Collabrate with Tiantian Zeng and Liyu Gong}{}
\begin{itemize}
  \item Use R to build function in inclass project to learn multivariate linear regression by write AIC, PressP, ridge  regularization. 
  \item Use github and Cran's R function to do automatically variable selection, find $R^{2}$ is enhanced from 0.2254 to 0.2802 with less selected variables, which is a great enhancement and also avoid overfitting.
\end{itemize}

% Reference Test
%\datedsubsection{\textbf{Paper Title\cite{zaharia2012resilient}}}{May. 2015}
%An xxx optimized for xxx\cite{verma2015large}
%\begin{itemize}
%  \item main contribution
%\end{itemize}


%% Reference
%\newpage
%\bibliographystyle{IEEETran}
%\bibliography{mycite}
\end{document}
