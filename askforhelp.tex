% !TEX program = xelatex

\documentclass{resume}
%\usepackage{zh_CN-Adobefonts_external} % Simplified Chinese Support using external fonts (./fonts/zh_CN-Adobe/)
%\usepackage{zh_CN-Adobefonts_internal} % Simplified Chinese Support using system fonts

\begin{document}
\pagenumbering{gobble} % suppress displaying page number

\name{Sheng Yuan}

\basicInfo{
  \email{} \textperiodcentered\ 
  \phone{} \ }

\section{\faGraduationCap\ Education}
\datedsubsection{\textbf{University of XXXX}, xxxx, U.S.}{2018 -- Present}
\textit{Ph.D.} in Statistics, expected May 2021
\datedsubsection{\textbf{University of XXXX}, XXXX, U.S.}{2016 -- 2018}
\textit{Master of Science} in Statistics
\datedsubsection{\textbf{XXXX University}, XXXX, China}{2011 -- 2015}
\textit{B.S.} in Statistics

\section{\faCogs\ Skills}
\begin{itemize}[parsep=0.5ex]
  \item Programming Languages: R == SAS == JMP> Python > SQL
  \item Data Skill: Big data analysis, Machine Learning, Deep Learning
\end{itemize}


\section{\faUsers\ Projects}
\datedsubsection{\textbf{Adding variability to Deep Neural Network}}{ July 2019 -- Present}
\role{Python, Pytorch, Tensorflow, Linux}{Individual Project}
\begin{itemize}
  \item Using Bayesian method (variationsl inference, etc.) to add variability on the parameters of deep neural networks, such as CNN and GAN models, and build 'confidence interval' like what we have done in statistic model. 
  \item Using Bayesian method in deep learning to avoid overfitting, and compare with other method like dropout or L1(L2) regularizations.
\end{itemize}

\datedsubsection{\textbf{111: Assessment Of Adult Non-trauma Icu Transfer Requests In Kentucky Using Publicly Available Data}}{May. 2018 -- Jan. 2019}
\role{Second Author}{Collabrate with xxxx}
Accepted by XXXXXXXXXXXX: January 2019 - Volume 47 - Issue 1 - p 38
\begin{itemize}
  \item Use Python to do data summary for Hospital Cost and Utilization Project (HCUP) data in XX (2011–16). (critical illness patients percentile, icu beds summary, etc.)
  \item Use the expected value in contigency table to analyze two categorical variables, and showed the trend. Using partial F test for quantitative data based on categorical variable, showed statistical significance.
  \item Compare critical illness(sepsis) to the whole dataset, showed sepsis patients that they will transferred to higher case volume hospitals, and had lower mortality rate. (confidence interval for sepsis patients far higher)
\end{itemize}

\datedsubsection{\textbf{Associations between changes in olfactory perception and cognitive symptoms in mild traumatic brain injury}}{May. 2018 -- }
\role{Second Author}{Collabrate with XXXXXXXX}
Submitted to XXXX XXXXXXXX, Jan.2019
\begin{itemize}
  \item Use JMP to do categorical data analysis (Fisher's exact test) on variables may be connected to olfactory perception's changes in brain injury, report p-value and find expressive difficulties, memory deficits are relatedto olfactory deficits.
  \item Use SAS to do logistic regression by backward-forward selection method to detect odds ratio's change on olfactory deficits and show results that like male are 24\% more likely to be in olfactory deficits and so on.
\end{itemize}

\datedsubsection{\textbf{Project: Determinants of Plasma Retinol and Beta-Carotene Levels
}}{March. 2017 -- May. 2017}
\role{Collabrate with xxxx xxxx and xxxx xxxx}{}
In class project to learn multivariate linear regression
\begin{itemize}
  \item Use R to do variable selection based on AIC and other  
  \item Compared our final model to CMU's model, and showed our analysis advantage
\end{itemize}

% Reference Test
%\datedsubsection{\textbf{Paper Title\cite{zaharia2012resilient}}}{May. 2015}
%An xxx optimized for xxx\cite{verma2015large}
%\begin{itemize}
%  \item main contribution
%\end{itemize}

\section{\faHeartO\ Honors and Awards}
\datedline{Teaching assistant scholarship, University of xxxx}{May. 2016 - Now}


%% Reference
%\newpage
%\bibliographystyle{IEEETran}
%\bibliography{mycite}
\end{document}
